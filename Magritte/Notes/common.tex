\documentclass[envcountsame,envcountchap]{svmono}

\usepackage{graphicx}
\usepackage{ifthen}
\usepackage{float}
\usepackage{longtable}
\usepackage{hyperref}
\usepackage{makeidx}
\usepackage{multicol}
\usepackage{theorem}
\usepackage{multirow}
\usepackage{tabularx}
\usepackage[bottom]{footmisc}
\usepackage{xspace}
\usepackage{alltt}
\usepackage{amssymb}
\usepackage[usenames,dvipsnames]{color}
\usepackage{colortbl}
\usepackage{bookman}
\usepackage{rotating}

\newboolean{complete}
\setboolean{complete}{false}

%% for this to happen start the file with
%\ifx\wholebook\relax\else
%\documentclass[envcountsame,envcountchap]{svmono}

\usepackage{graphicx}
\usepackage{ifthen}
\usepackage{float}
\usepackage{longtable}
\usepackage{hyperref}
\usepackage{makeidx}
\usepackage{multicol}
\usepackage{theorem}
\usepackage{multirow}
\usepackage{tabularx}
\usepackage[bottom]{footmisc}
\usepackage{xspace}
\usepackage{alltt}
\usepackage{amssymb}
\usepackage[usenames,dvipsnames]{color}
\usepackage{colortbl}
\usepackage{bookman}
\usepackage{rotating}

\newboolean{complete}
\setboolean{complete}{false}

%% for this to happen start the file with
%\ifx\wholebook\relax\else
%\documentclass[envcountsame,envcountchap]{svmono}

\usepackage{graphicx}
\usepackage{ifthen}
\usepackage{float}
\usepackage{longtable}
\usepackage{hyperref}
\usepackage{makeidx}
\usepackage{multicol}
\usepackage{theorem}
\usepackage{multirow}
\usepackage{tabularx}
\usepackage[bottom]{footmisc}
\usepackage{xspace}
\usepackage{alltt}
\usepackage{amssymb}
\usepackage[usenames,dvipsnames]{color}
\usepackage{colortbl}
\usepackage{bookman}
\usepackage{rotating}

\newboolean{complete}
\setboolean{complete}{false}

%% for this to happen start the file with
%\ifx\wholebook\relax\else
%\documentclass[envcountsame,envcountchap]{svmono}

\usepackage{graphicx}
\usepackage{ifthen}
\usepackage{float}
\usepackage{longtable}
\usepackage{hyperref}
\usepackage{makeidx}
\usepackage{multicol}
\usepackage{theorem}
\usepackage{multirow}
\usepackage{tabularx}
\usepackage[bottom]{footmisc}
\usepackage{xspace}
\usepackage{alltt}
\usepackage{amssymb}
\usepackage[usenames,dvipsnames]{color}
\usepackage{colortbl}
\usepackage{bookman}
\usepackage{rotating}

\newboolean{complete}
\setboolean{complete}{false}

%% for this to happen start the file with
%\ifx\wholebook\relax\else
%\input{../common.tex}
%\begin{document}
%\fi
% and terminate by
% \ifx\wholebook\relax\else\end{document}\fi

%\newif\ifpdf
%\ifx\pdfoutput\undefined
%\pdffalse % we are not running PDFLaTeX
%\else
%\pdfoutput=1 % we are running PDFLaTeX
%\pdftrue
%\fi

%\ifpdf
%\DeclareGraphicsExtensions{.pdf, .jpg, .png}
%\else
%\DeclareGraphicsExtensions{.eps, .jpg}
%\fi

\definecolor{codebackground}{rgb}{0.95,0.95,0.95} 

\newdimen\tmpdim
\def\begcbox #1{\def\coloring{#1}\setbox0\hbox\bgroup\tmpdim=\textwidth\advance\tmpdim by-2\fboxsep\begin{minipage}{\tmpdim}}
\def\endcbox{\end{minipage}\egroup\par\noindent{\colorbox{\coloring}{\box0}}}

\def\CodeBox{\begcbox{codebackground}}
\def\EndCodeBox{\endcbox}



\newcommand{\ie}{\textit{i.e.,}\xspace}
\newcommand{\eg}{\textit{e.g.,}\xspace}
\newcommand{\etc}{\textit{etc.}\xspace}

\newcommand{\secref}[1]{Section~\ref{#1}\xspace}
\newcommand{\figref}[1]{Figure~\ref{#1}\xspace}
\newcommand{\charef}[1]{Chapter~\ref{#1}\xspace}
\newcommand{\appref}[1]{Appendix~\ref{#1}\xspace}
\newcommand{\tabref}[1]{Table~\ref{#1}\xspace}

%%%for code in text
\newcommand{\ct}[1]{{\small\textsf{#1}}}

\newcommand{\nb}[2]{
    \fbox{\bfseries\sffamily\scriptsize#1}
    {\sf\small$\blacktriangleright$\textit{#2}$\blacktriangleleft$}
   }
\newcommand{\pav}[1]{\nb{pavel}{#1}}
\newcommand\stef[1]{\nb{SD}{#1}}

\newcommand{\fullrow}[1]{\multicolumn{8}{|>{\columncolor{yellow}}l|}{\sf {#1}}\\ \hline \hline}
\newcommand{\rowseven}[1]{\multicolumn{7}{p{9cm}|}{\sf {#1}}\\ \hline \hline}

\newcommand{\ycell}[1]{\multicolumn{1}{|>{\columncolor{yellow}}l|}{\sf #1}}
\newcommand{\defrow}[1]{\ycell{Definition} & \rowseven{#1}}
\newcommand{\yccell}[1]{\multicolumn{1}{>{\columncolor{yellow}}c|}{\sf #1}}
\newcommand{\wcell}[1]{\multicolumn{1}{|c|}{\sf #1}}
\newcommand{\widecell}[1]{\multicolumn{1}{p{2.5cm}|}{\sf #1}}

\newcommand{\wccell}[1]{\multicolumn{1}{c|}{\sf #1}}
\newcommand{\emptyfour}{\multicolumn{4}{l|}{}\\}
\newcommand{\emptytwo}{\multicolumn{2}{l|}{}\\}

\newenvironment{sfalltt}
    {\begin{scriptsize}\begin{alltt}\sffamily}
    {\end{alltt}\end{scriptsize}}

\setcounter{secnumdepth}{2}

\definecolor{darkred}{rgb}{0.5,0,0}
\definecolor{darkgreen}{rgb}{0,0.5,0}
\definecolor{darkblue}{rgb}{0,0,0.5}

\newboolean{isComplete}
\setboolean{isComplete}{false}

\newcommand{\printTWO}{\ifthenelse{\boolean{isComplete}}
       {\begin{quote}{\em IS COMPLETE}\end{quote}}
       {\begin{quote}{\em IS PARTIAL}\end{quote}}}


\newcommand{\completeOrPartial}[2]{\ifthenelse{\boolean{isComplete}}
       {\input{#1}}
       {\input{#2}}}

\setlength{\marginparsep}{2mm}
\newcommand{\marge}[1]{\marginpar{{\sffamily \bfseries #1}}}
\newcommand{\intent}{\marge{\hfill Description}}
\newcommand{\applies}{\marge{\hfill Applies To}}
\newcommand{\motiv}{\marge{\hfill Impact}}
\newcommand{\derule}{\marge{\hfill Detection}}
\newcommand{\heur}{\marge{\hfill Heuristics}}
\newcommand{\exam}{\marge{\hfill Example}}
\newcommand{\refac}{\marge{\hfill Refactoring}}
\newcommand{\remarks}{\marge{\hfill Remarks}}
\newcommand{\skiphalf}{\vspace{0.5cm}}
\newcommand{\skipquarter}{\vspace{0.25cm}}
\newcommand{\dstitle}[1]{\section{{#1}}\rule{\textwidth}{0.05cm} \skipquarter}
\newcommand{\chatitle}[1]{\chapter{{#1}}}
\newcommand{\sectitle}[1]{\section{{#1}}}
\newcommand{\subsectitle}[1]{\subsection{{#1}}}
\newcommand{\subsubsectitle}[1]{\subsubsection{{#1}}}

\renewcommand{\baselinestretch}{1.1}

%%% here we should add the ffolders where the figures should be looked up
\graphicspath{{./SEA1/}{./SEA2}}

\hypersetup{
   a4paper,
   pdfstartview=FitV,
   colorlinks,
   linkcolor=darkblue,
   citecolor=darkblue,
   pdftitle={Getting Started with Seaside},
   pdfauthor={Pavel Krivanek}
}

%GATHER{bib/scg.bib}
%GATHER{oomip.bbl}


%\newenvironment{code}
%{\small \par\sf
%\begin{tabbing}
%xxxx\=xxxx\=xxxx\=xxxx\=xxxx\=xxxxxxxxxxxxxxxxxxxxxxxxxxxxxxxxxxxxx\=
%xxxx\=xxxx\=xxxx\=xxxx\=xxxx\=\kill}
%{\end{tabbing}
%\sf\normalsize
%\par
%}    

\newenvironment{code}
{\small\par\begin{alltt}}
{\end{alltt}\sf\normalsize
\par}    

\newcommand{\sep}{$>>$}
\newcommand{\pipe}{$|$}

\newenvironment{bcode}
    {\begin{alltt}\CodeBox\sffamily\begin{small}}
    {\end{small}\EndCodeBox\end{alltt}}



%\begin{document}
%\fi
% and terminate by
% \ifx\wholebook\relax\else\end{document}\fi

%\newif\ifpdf
%\ifx\pdfoutput\undefined
%\pdffalse % we are not running PDFLaTeX
%\else
%\pdfoutput=1 % we are running PDFLaTeX
%\pdftrue
%\fi

%\ifpdf
%\DeclareGraphicsExtensions{.pdf, .jpg, .png}
%\else
%\DeclareGraphicsExtensions{.eps, .jpg}
%\fi

\definecolor{codebackground}{rgb}{0.95,0.95,0.95} 

\newdimen\tmpdim
\def\begcbox #1{\def\coloring{#1}\setbox0\hbox\bgroup\tmpdim=\textwidth\advance\tmpdim by-2\fboxsep\begin{minipage}{\tmpdim}}
\def\endcbox{\end{minipage}\egroup\par\noindent{\colorbox{\coloring}{\box0}}}

\def\CodeBox{\begcbox{codebackground}}
\def\EndCodeBox{\endcbox}



\newcommand{\ie}{\textit{i.e.,}\xspace}
\newcommand{\eg}{\textit{e.g.,}\xspace}
\newcommand{\etc}{\textit{etc.}\xspace}

\newcommand{\secref}[1]{Section~\ref{#1}\xspace}
\newcommand{\figref}[1]{Figure~\ref{#1}\xspace}
\newcommand{\charef}[1]{Chapter~\ref{#1}\xspace}
\newcommand{\appref}[1]{Appendix~\ref{#1}\xspace}
\newcommand{\tabref}[1]{Table~\ref{#1}\xspace}

%%%for code in text
\newcommand{\ct}[1]{{\small\textsf{#1}}}

\newcommand{\nb}[2]{
    \fbox{\bfseries\sffamily\scriptsize#1}
    {\sf\small$\blacktriangleright$\textit{#2}$\blacktriangleleft$}
   }
\newcommand{\pav}[1]{\nb{pavel}{#1}}
\newcommand\stef[1]{\nb{SD}{#1}}

\newcommand{\fullrow}[1]{\multicolumn{8}{|>{\columncolor{yellow}}l|}{\sf {#1}}\\ \hline \hline}
\newcommand{\rowseven}[1]{\multicolumn{7}{p{9cm}|}{\sf {#1}}\\ \hline \hline}

\newcommand{\ycell}[1]{\multicolumn{1}{|>{\columncolor{yellow}}l|}{\sf #1}}
\newcommand{\defrow}[1]{\ycell{Definition} & \rowseven{#1}}
\newcommand{\yccell}[1]{\multicolumn{1}{>{\columncolor{yellow}}c|}{\sf #1}}
\newcommand{\wcell}[1]{\multicolumn{1}{|c|}{\sf #1}}
\newcommand{\widecell}[1]{\multicolumn{1}{p{2.5cm}|}{\sf #1}}

\newcommand{\wccell}[1]{\multicolumn{1}{c|}{\sf #1}}
\newcommand{\emptyfour}{\multicolumn{4}{l|}{}\\}
\newcommand{\emptytwo}{\multicolumn{2}{l|}{}\\}

\newenvironment{sfalltt}
    {\begin{scriptsize}\begin{alltt}\sffamily}
    {\end{alltt}\end{scriptsize}}

\setcounter{secnumdepth}{2}

\definecolor{darkred}{rgb}{0.5,0,0}
\definecolor{darkgreen}{rgb}{0,0.5,0}
\definecolor{darkblue}{rgb}{0,0,0.5}

\newboolean{isComplete}
\setboolean{isComplete}{false}

\newcommand{\printTWO}{\ifthenelse{\boolean{isComplete}}
       {\begin{quote}{\em IS COMPLETE}\end{quote}}
       {\begin{quote}{\em IS PARTIAL}\end{quote}}}


\newcommand{\completeOrPartial}[2]{\ifthenelse{\boolean{isComplete}}
       {\input{#1}}
       {\input{#2}}}

\setlength{\marginparsep}{2mm}
\newcommand{\marge}[1]{\marginpar{{\sffamily \bfseries #1}}}
\newcommand{\intent}{\marge{\hfill Description}}
\newcommand{\applies}{\marge{\hfill Applies To}}
\newcommand{\motiv}{\marge{\hfill Impact}}
\newcommand{\derule}{\marge{\hfill Detection}}
\newcommand{\heur}{\marge{\hfill Heuristics}}
\newcommand{\exam}{\marge{\hfill Example}}
\newcommand{\refac}{\marge{\hfill Refactoring}}
\newcommand{\remarks}{\marge{\hfill Remarks}}
\newcommand{\skiphalf}{\vspace{0.5cm}}
\newcommand{\skipquarter}{\vspace{0.25cm}}
\newcommand{\dstitle}[1]{\section{{#1}}\rule{\textwidth}{0.05cm} \skipquarter}
\newcommand{\chatitle}[1]{\chapter{{#1}}}
\newcommand{\sectitle}[1]{\section{{#1}}}
\newcommand{\subsectitle}[1]{\subsection{{#1}}}
\newcommand{\subsubsectitle}[1]{\subsubsection{{#1}}}

\renewcommand{\baselinestretch}{1.1}

%%% here we should add the ffolders where the figures should be looked up
\graphicspath{{./SEA1/}{./SEA2}}

\hypersetup{
   a4paper,
   pdfstartview=FitV,
   colorlinks,
   linkcolor=darkblue,
   citecolor=darkblue,
   pdftitle={Getting Started with Seaside},
   pdfauthor={Pavel Krivanek}
}

%GATHER{bib/scg.bib}
%GATHER{oomip.bbl}


%\newenvironment{code}
%{\small \par\sf
%\begin{tabbing}
%xxxx\=xxxx\=xxxx\=xxxx\=xxxx\=xxxxxxxxxxxxxxxxxxxxxxxxxxxxxxxxxxxxx\=
%xxxx\=xxxx\=xxxx\=xxxx\=xxxx\=\kill}
%{\end{tabbing}
%\sf\normalsize
%\par
%}    

\newenvironment{code}
{\small\par\begin{alltt}}
{\end{alltt}\sf\normalsize
\par}    

\newcommand{\sep}{$>>$}
\newcommand{\pipe}{$|$}

\newenvironment{bcode}
    {\begin{alltt}\CodeBox\sffamily\begin{small}}
    {\end{small}\EndCodeBox\end{alltt}}



%\begin{document}
%\fi
% and terminate by
% \ifx\wholebook\relax\else\end{document}\fi

%\newif\ifpdf
%\ifx\pdfoutput\undefined
%\pdffalse % we are not running PDFLaTeX
%\else
%\pdfoutput=1 % we are running PDFLaTeX
%\pdftrue
%\fi

%\ifpdf
%\DeclareGraphicsExtensions{.pdf, .jpg, .png}
%\else
%\DeclareGraphicsExtensions{.eps, .jpg}
%\fi

\definecolor{codebackground}{rgb}{0.95,0.95,0.95} 

\newdimen\tmpdim
\def\begcbox #1{\def\coloring{#1}\setbox0\hbox\bgroup\tmpdim=\textwidth\advance\tmpdim by-2\fboxsep\begin{minipage}{\tmpdim}}
\def\endcbox{\end{minipage}\egroup\par\noindent{\colorbox{\coloring}{\box0}}}

\def\CodeBox{\begcbox{codebackground}}
\def\EndCodeBox{\endcbox}



\newcommand{\ie}{\textit{i.e.,}\xspace}
\newcommand{\eg}{\textit{e.g.,}\xspace}
\newcommand{\etc}{\textit{etc.}\xspace}

\newcommand{\secref}[1]{Section~\ref{#1}\xspace}
\newcommand{\figref}[1]{Figure~\ref{#1}\xspace}
\newcommand{\charef}[1]{Chapter~\ref{#1}\xspace}
\newcommand{\appref}[1]{Appendix~\ref{#1}\xspace}
\newcommand{\tabref}[1]{Table~\ref{#1}\xspace}

%%%for code in text
\newcommand{\ct}[1]{{\small\textsf{#1}}}

\newcommand{\nb}[2]{
    \fbox{\bfseries\sffamily\scriptsize#1}
    {\sf\small$\blacktriangleright$\textit{#2}$\blacktriangleleft$}
   }
\newcommand{\pav}[1]{\nb{pavel}{#1}}
\newcommand\stef[1]{\nb{SD}{#1}}

\newcommand{\fullrow}[1]{\multicolumn{8}{|>{\columncolor{yellow}}l|}{\sf {#1}}\\ \hline \hline}
\newcommand{\rowseven}[1]{\multicolumn{7}{p{9cm}|}{\sf {#1}}\\ \hline \hline}

\newcommand{\ycell}[1]{\multicolumn{1}{|>{\columncolor{yellow}}l|}{\sf #1}}
\newcommand{\defrow}[1]{\ycell{Definition} & \rowseven{#1}}
\newcommand{\yccell}[1]{\multicolumn{1}{>{\columncolor{yellow}}c|}{\sf #1}}
\newcommand{\wcell}[1]{\multicolumn{1}{|c|}{\sf #1}}
\newcommand{\widecell}[1]{\multicolumn{1}{p{2.5cm}|}{\sf #1}}

\newcommand{\wccell}[1]{\multicolumn{1}{c|}{\sf #1}}
\newcommand{\emptyfour}{\multicolumn{4}{l|}{}\\}
\newcommand{\emptytwo}{\multicolumn{2}{l|}{}\\}

\newenvironment{sfalltt}
    {\begin{scriptsize}\begin{alltt}\sffamily}
    {\end{alltt}\end{scriptsize}}

\setcounter{secnumdepth}{2}

\definecolor{darkred}{rgb}{0.5,0,0}
\definecolor{darkgreen}{rgb}{0,0.5,0}
\definecolor{darkblue}{rgb}{0,0,0.5}

\newboolean{isComplete}
\setboolean{isComplete}{false}

\newcommand{\printTWO}{\ifthenelse{\boolean{isComplete}}
       {\begin{quote}{\em IS COMPLETE}\end{quote}}
       {\begin{quote}{\em IS PARTIAL}\end{quote}}}


\newcommand{\completeOrPartial}[2]{\ifthenelse{\boolean{isComplete}}
       {\input{#1}}
       {\input{#2}}}

\setlength{\marginparsep}{2mm}
\newcommand{\marge}[1]{\marginpar{{\sffamily \bfseries #1}}}
\newcommand{\intent}{\marge{\hfill Description}}
\newcommand{\applies}{\marge{\hfill Applies To}}
\newcommand{\motiv}{\marge{\hfill Impact}}
\newcommand{\derule}{\marge{\hfill Detection}}
\newcommand{\heur}{\marge{\hfill Heuristics}}
\newcommand{\exam}{\marge{\hfill Example}}
\newcommand{\refac}{\marge{\hfill Refactoring}}
\newcommand{\remarks}{\marge{\hfill Remarks}}
\newcommand{\skiphalf}{\vspace{0.5cm}}
\newcommand{\skipquarter}{\vspace{0.25cm}}
\newcommand{\dstitle}[1]{\section{{#1}}\rule{\textwidth}{0.05cm} \skipquarter}
\newcommand{\chatitle}[1]{\chapter{{#1}}}
\newcommand{\sectitle}[1]{\section{{#1}}}
\newcommand{\subsectitle}[1]{\subsection{{#1}}}
\newcommand{\subsubsectitle}[1]{\subsubsection{{#1}}}

\renewcommand{\baselinestretch}{1.1}

%%% here we should add the ffolders where the figures should be looked up
\graphicspath{{./SEA1/}{./SEA2}}

\hypersetup{
   a4paper,
   pdfstartview=FitV,
   colorlinks,
   linkcolor=darkblue,
   citecolor=darkblue,
   pdftitle={Getting Started with Seaside},
   pdfauthor={Pavel Krivanek}
}

%GATHER{bib/scg.bib}
%GATHER{oomip.bbl}


%\newenvironment{code}
%{\small \par\sf
%\begin{tabbing}
%xxxx\=xxxx\=xxxx\=xxxx\=xxxx\=xxxxxxxxxxxxxxxxxxxxxxxxxxxxxxxxxxxxx\=
%xxxx\=xxxx\=xxxx\=xxxx\=xxxx\=\kill}
%{\end{tabbing}
%\sf\normalsize
%\par
%}    

\newenvironment{code}
{\small\par\begin{alltt}}
{\end{alltt}\sf\normalsize
\par}    

\newcommand{\sep}{$>>$}
\newcommand{\pipe}{$|$}

\newenvironment{bcode}
    {\begin{alltt}\CodeBox\sffamily\begin{small}}
    {\end{small}\EndCodeBox\end{alltt}}



%\begin{document}
%\fi
% and terminate by
% \ifx\wholebook\relax\else\end{document}\fi

%\newif\ifpdf
%\ifx\pdfoutput\undefined
%\pdffalse % we are not running PDFLaTeX
%\else
%\pdfoutput=1 % we are running PDFLaTeX
%\pdftrue
%\fi

%\ifpdf
%\DeclareGraphicsExtensions{.pdf, .jpg, .png}
%\else
%\DeclareGraphicsExtensions{.eps, .jpg}
%\fi

\definecolor{codebackground}{rgb}{0.95,0.95,0.95} 

\newdimen\tmpdim
\def\begcbox #1{\def\coloring{#1}\setbox0\hbox\bgroup\tmpdim=\textwidth\advance\tmpdim by-2\fboxsep\begin{minipage}{\tmpdim}}
\def\endcbox{\end{minipage}\egroup\par\noindent{\colorbox{\coloring}{\box0}}}

\def\CodeBox{\begcbox{codebackground}}
\def\EndCodeBox{\endcbox}



\newcommand{\ie}{\textit{i.e.,}\xspace}
\newcommand{\eg}{\textit{e.g.,}\xspace}
\newcommand{\etc}{\textit{etc.}\xspace}

\newcommand{\secref}[1]{Section~\ref{#1}\xspace}
\newcommand{\figref}[1]{Figure~\ref{#1}\xspace}
\newcommand{\charef}[1]{Chapter~\ref{#1}\xspace}
\newcommand{\appref}[1]{Appendix~\ref{#1}\xspace}
\newcommand{\tabref}[1]{Table~\ref{#1}\xspace}

%%%for code in text
\newcommand{\ct}[1]{{\small\textsf{#1}}}

\newcommand{\nb}[2]{
    \fbox{\bfseries\sffamily\scriptsize#1}
    {\sf\small$\blacktriangleright$\textit{#2}$\blacktriangleleft$}
   }
\newcommand{\pav}[1]{\nb{pavel}{#1}}
\newcommand\stef[1]{\nb{SD}{#1}}

\newcommand{\fullrow}[1]{\multicolumn{8}{|>{\columncolor{yellow}}l|}{\sf {#1}}\\ \hline \hline}
\newcommand{\rowseven}[1]{\multicolumn{7}{p{9cm}|}{\sf {#1}}\\ \hline \hline}

\newcommand{\ycell}[1]{\multicolumn{1}{|>{\columncolor{yellow}}l|}{\sf #1}}
\newcommand{\defrow}[1]{\ycell{Definition} & \rowseven{#1}}
\newcommand{\yccell}[1]{\multicolumn{1}{>{\columncolor{yellow}}c|}{\sf #1}}
\newcommand{\wcell}[1]{\multicolumn{1}{|c|}{\sf #1}}
\newcommand{\widecell}[1]{\multicolumn{1}{p{2.5cm}|}{\sf #1}}

\newcommand{\wccell}[1]{\multicolumn{1}{c|}{\sf #1}}
\newcommand{\emptyfour}{\multicolumn{4}{l|}{}\\}
\newcommand{\emptytwo}{\multicolumn{2}{l|}{}\\}

\newenvironment{sfalltt}
    {\begin{scriptsize}\begin{alltt}\sffamily}
    {\end{alltt}\end{scriptsize}}

\setcounter{secnumdepth}{2}

\definecolor{darkred}{rgb}{0.5,0,0}
\definecolor{darkgreen}{rgb}{0,0.5,0}
\definecolor{darkblue}{rgb}{0,0,0.5}

\newboolean{isComplete}
\setboolean{isComplete}{false}

\newcommand{\printTWO}{\ifthenelse{\boolean{isComplete}}
       {\begin{quote}{\em IS COMPLETE}\end{quote}}
       {\begin{quote}{\em IS PARTIAL}\end{quote}}}


\newcommand{\completeOrPartial}[2]{\ifthenelse{\boolean{isComplete}}
       {\input{#1}}
       {\input{#2}}}

\setlength{\marginparsep}{2mm}
\newcommand{\marge}[1]{\marginpar{{\sffamily \bfseries #1}}}
\newcommand{\intent}{\marge{\hfill Description}}
\newcommand{\applies}{\marge{\hfill Applies To}}
\newcommand{\motiv}{\marge{\hfill Impact}}
\newcommand{\derule}{\marge{\hfill Detection}}
\newcommand{\heur}{\marge{\hfill Heuristics}}
\newcommand{\exam}{\marge{\hfill Example}}
\newcommand{\refac}{\marge{\hfill Refactoring}}
\newcommand{\remarks}{\marge{\hfill Remarks}}
\newcommand{\skiphalf}{\vspace{0.5cm}}
\newcommand{\skipquarter}{\vspace{0.25cm}}
\newcommand{\dstitle}[1]{\section{{#1}}\rule{\textwidth}{0.05cm} \skipquarter}
\newcommand{\chatitle}[1]{\chapter{{#1}}}
\newcommand{\sectitle}[1]{\section{{#1}}}
\newcommand{\subsectitle}[1]{\subsection{{#1}}}
\newcommand{\subsubsectitle}[1]{\subsubsection{{#1}}}

\renewcommand{\baselinestretch}{1.1}

%%% here we should add the ffolders where the figures should be looked up
\graphicspath{{./SEA1/}{./SEA2}}

\hypersetup{
   a4paper,
   pdfstartview=FitV,
   colorlinks,
   linkcolor=darkblue,
   citecolor=darkblue,
   pdftitle={Getting Started with Seaside},
   pdfauthor={Pavel Krivanek}
}

%GATHER{bib/scg.bib}
%GATHER{oomip.bbl}


%\newenvironment{code}
%{\small \par\sf
%\begin{tabbing}
%xxxx\=xxxx\=xxxx\=xxxx\=xxxx\=xxxxxxxxxxxxxxxxxxxxxxxxxxxxxxxxxxxxx\=
%xxxx\=xxxx\=xxxx\=xxxx\=xxxx\=\kill}
%{\end{tabbing}
%\sf\normalsize
%\par
%}    

\newenvironment{code}
{\small\par\begin{alltt}}
{\end{alltt}\sf\normalsize
\par}    

\newcommand{\sep}{$>>$}
\newcommand{\pipe}{$|$}

\newenvironment{bcode}
    {\begin{alltt}\CodeBox\sffamily\begin{small}}
    {\end{small}\EndCodeBox\end{alltt}}


